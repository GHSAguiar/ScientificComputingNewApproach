\documentclass{article}
\usepackage[utf8]{inputenc}
\usepackage{amsmath}
\usepackage{amsfonts}
\usepackage{graphicx}
\graphicspath{ {./images/} }

\title{Machine Learning}
\author{Gabriel Aguiar}
\date{November 2019}

\begin{document}

\maketitle

\section{A brief historical overview}

\hfill

It is very difficult to pinpoint the birth of what we know today as machine learning. However, it is possible to begin to approach its history through the ideas of Warren McCulloch and Walter Pitss, in 1943.

\hfill

The first character was an american neuroanatomist and psychiatrist, while Pitts was involved in work in the field of mathematical logic. From the intertwining of their respective areas, these researchers devised the first "artificial neuron".

\hfill

This neuron is capable of receiving distinct inputs, weighted in different ways. Thus, in its structure there is a compelling summing signal and then an activation function responsible for an output.

\hfill

In 1948, Alan Turing made immeasurable contributions to the concept of machine learning. His works range from a theoretical development to the construction of functional machines.

\hfill

In 1957, Rosenblatt's Perceptron, the first pattern recognition algorithm, appeared. With this, Frank Rosenblatt opened a window of hope for the conception that machines can "learn".

\hfill

In 1959, Arthur Samuel put in history the term "machine learning". His work focused on machine training using the Game of Checkers.

\hfill

In 1969 Marvin Minsky and Seymour Papert wrote a book that undermines all hope introduced by Rosenblatt's Perceptron. In short, Minsky and Papert indicated an extremely basic logic gate that could not be understood by Perceptron: the XOR gate.

\hfill

Amid disbelief about machine learning, the 1980s brought back its strength. In 1982, John Hopfield built one of the first "artificial neural networks". In 1986, David Rumelhart, Geoffrey Hintont and Ronald Williams introduced the notion of "backpropagation", a learning algorithm widely employed to this day.

\hfill

It is said that the 1980s symbolized the revival of machine learning and that the 1990s represented its conceptual maturation.

\hfill

In this way, the 21st century has already begun with a strong trend in machine learning. The 2000s were marked by the consolidation of Data Science, as well as improvements in hardware and tools. In the 2010s, major builds took place in modules, Cloud computing, Internet of Things (IoT) and Generative Adversarial Networks (GANs).

\end{document}