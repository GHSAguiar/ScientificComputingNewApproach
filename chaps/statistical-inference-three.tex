\documentclass{article}
\usepackage[utf8]{inputenc}
\usepackage{amsmath}
\usepackage{amsfonts}

\title{Statistical inference}
\author{Gabriel Aguiar}
\date{September 2019}

\begin{document}

\maketitle

One of the possible approaches to statistical inference lies in a method known as regression. Such methodology aims to fit a function to a data set. Here, we will see how all this arises from the Bayes' theorem.

Take a data set that represents a certain event. If we want to find a function parameter that best fits this set, we can make use of the probability theory:

\hfill

$P(\theta/D,I) = \frac{1}{\sum\limits_{\theta_{i}} P(D/\theta_{i},I) \; P(\theta_{i})/I} \; P(D/\theta,I) \; P(\theta/I)$

\hfill

$\theta$: Parameter; $D$: Data; $I$: Context information

\hfill

We can imagine that the parameter obeys a normal distribution of mean $\mu$ and standard deviation $\sigma$, since for random errors around it, there is an equal probability of getting larger or smaller values:

\hfill

$P(D/\theta,I) = \frac{1}{\sqrt{2 \pi \sigma^{2}}} \; e^{\frac{(\theta - \mu)^{2}}{2 \sigma^{2}}}$

\hfill

By the Principle of sufficient reason, we have no cause to point out a bias towards $\theta$. In this way we can express the function $P(\theta/I)$ through the following behavior:

\hfill

$P(\theta/I) = c$, $0 \leq \theta \leq 1$, with $c$ constant

\hfill

Thus, as the function $P(\theta/D,I)$ denominator also represents a constant, we have:

\hfill

$P(\theta/D,I) = C \frac{1}{\sqrt{2 \pi \sigma^{2}}} \; e^{-\frac{(\theta - \mu)^{2}}{2 \sigma^{2}}}$, with $C$ constant

\hfill

If we take the natural logarithm of the function $P(\theta/D,I)$:

\hfill

$L(\theta/D,I) \equiv ln \; P(\theta/D,I) = ln \; C - \frac{1}{2} \; ln \; 2 \pi \sigma^{2} \; -\frac{1}{2} \; \frac{(\theta - \mu)^{2}}{2 \sigma^{2}}$

\hfill

So if we want to maximize $P(\theta/D,I)$, we need to minimize $\frac{(\theta - \mu)^{2}}{\sigma^{2}}$.

\hfill

In statistics, this term is known as chi square and symbolizes the regression methodology:

\hfill

$\chi^{2} \equiv \frac{(\theta - \mu)^{2}}{\sigma^{2}}$

\end{document}