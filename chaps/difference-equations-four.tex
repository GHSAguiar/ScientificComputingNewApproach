\documentclass{article}
\usepackage[utf8]{inputenc}
\usepackage{amsmath}
\usepackage{amsfonts}
\usepackage{graphicx}
\graphicspath{ {./images/} }

\title{Difference equations}
\author{Gabriel Aguiar}
\date{October 2019}

\begin{document}

\maketitle

\section{Iterative methods in numerical solution of equations}

Consider the following problem: Given a function $f$, find the values of $x$ such that $f(x) = 0$. Let's build a function $\phi$, which we will call a "map", satisfying:

\begin{itemize}
    
\item $\phi (x_{k}) = x_{k + 1}$

Thus, given an $x_{k}$ (associated with iteration $k$), the map function returns the later value $x_{k + 1}$.

\item $f(x^{*}) = 0 \Rightarrow \phi (x^{*}) = x^{*}$

$x^{*}$ is called the fixed point of $\phi$.

\end{itemize}

Let's look at the following example: $f(x) = x^{2} + 0.96 \; x - 2.08$

\hfill

A possible construct for $\phi$ is: $\phi (x) = \frac{2.08}{x_{k} + 0.96}$

\hfill

At your fixed point: $x^{*} = \frac{2.08}{x^{*} + 0.96} \Rightarrow (x^{*})^{2} + 0.96 \; x^{*} - 2.08 = 0$

\hfill

Let's now look at some formulations for addressing problems like this:

\begin{itemize}

\item Brouwer's fixed point theorem

Suppose $\phi \in C[a,b]$, such that $\phi (x) \in [a,b], \forall \; x \in [a,b]$. So $\exists \; x^{*} \in [a,b]$ such that $\phi (x^{*}) = x^{*}$.

\item Contraction definition

$\phi$ is a contraction in $[a,b]$ if there is $0 < L < 1$ such that:

$| \phi (x) - \phi (y) | \leq L \; | x - y |$, $\forall \; x,y \in [a,b]$

$\phi$ is said to be $L$-Lipschitz.

\item Contraction map theorem

Let F be under conditions of the Brouwer's fixed point theorem. If $\phi$ is a contraction in $[a,b]$, then the fixed point $x^{*}$ will be unique. Additionally:

$lim_{k \rightarrow \infty} \; \phi (x_{k}) = x^{*}$, $\forall \; x_{0} \in [a,b]$

$x_{0}$ is said to be the start point.

\item Theorem

Given $x_{0} \in [a,b]$ and a tolerance $\epsilon > 0$:

$k(\epsilon) > \frac{1}{log (\frac{1}{L})} \; log \; | x_{1} - x_{0} | - log \; (\epsilon \; (1 - L))$

$\phi (x_{k}) = x_{k + 1}$ an $L$-contraction.

\end{itemize}

In this context, we can introduce the so-called Relaxation method, in which $\phi$ has the following feature: $\phi (x) = x - \lambda (x) \; f(x)$.

\hfill

In a specific case of $\lambda (x)$, we have the so-called Newton-Raphson method:

\hfill

$\phi (x_{k}) = x_{k + 1} = x_{k} - \frac{f(x_{k})}{f'(x_{k})}$

\end{document}